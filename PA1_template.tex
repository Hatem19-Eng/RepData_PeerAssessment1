\PassOptionsToPackage{unicode=true}{hyperref} % options for packages loaded elsewhere
\PassOptionsToPackage{hyphens}{url}
%
\documentclass[]{article}
\usepackage{lmodern}
\usepackage{amssymb,amsmath}
\usepackage{ifxetex,ifluatex}
\usepackage{fixltx2e} % provides \textsubscript
\ifnum 0\ifxetex 1\fi\ifluatex 1\fi=0 % if pdftex
  \usepackage[T1]{fontenc}
  \usepackage[utf8]{inputenc}
  \usepackage{textcomp} % provides euro and other symbols
\else % if luatex or xelatex
  \usepackage{unicode-math}
  \defaultfontfeatures{Ligatures=TeX,Scale=MatchLowercase}
\fi
% use upquote if available, for straight quotes in verbatim environments
\IfFileExists{upquote.sty}{\usepackage{upquote}}{}
% use microtype if available
\IfFileExists{microtype.sty}{%
\usepackage[]{microtype}
\UseMicrotypeSet[protrusion]{basicmath} % disable protrusion for tt fonts
}{}
\IfFileExists{parskip.sty}{%
\usepackage{parskip}
}{% else
\setlength{\parindent}{0pt}
\setlength{\parskip}{6pt plus 2pt minus 1pt}
}
\usepackage{hyperref}
\hypersetup{
            pdftitle={Repreducible Reasearch Course Project 1},
            pdfborder={0 0 0},
            breaklinks=true}
\urlstyle{same}  % don't use monospace font for urls
\usepackage[margin=1in]{geometry}
\usepackage{color}
\usepackage{fancyvrb}
\newcommand{\VerbBar}{|}
\newcommand{\VERB}{\Verb[commandchars=\\\{\}]}
\DefineVerbatimEnvironment{Highlighting}{Verbatim}{commandchars=\\\{\}}
% Add ',fontsize=\small' for more characters per line
\usepackage{framed}
\definecolor{shadecolor}{RGB}{248,248,248}
\newenvironment{Shaded}{\begin{snugshade}}{\end{snugshade}}
\newcommand{\AlertTok}[1]{\textcolor[rgb]{0.94,0.16,0.16}{#1}}
\newcommand{\AnnotationTok}[1]{\textcolor[rgb]{0.56,0.35,0.01}{\textbf{\textit{#1}}}}
\newcommand{\AttributeTok}[1]{\textcolor[rgb]{0.77,0.63,0.00}{#1}}
\newcommand{\BaseNTok}[1]{\textcolor[rgb]{0.00,0.00,0.81}{#1}}
\newcommand{\BuiltInTok}[1]{#1}
\newcommand{\CharTok}[1]{\textcolor[rgb]{0.31,0.60,0.02}{#1}}
\newcommand{\CommentTok}[1]{\textcolor[rgb]{0.56,0.35,0.01}{\textit{#1}}}
\newcommand{\CommentVarTok}[1]{\textcolor[rgb]{0.56,0.35,0.01}{\textbf{\textit{#1}}}}
\newcommand{\ConstantTok}[1]{\textcolor[rgb]{0.00,0.00,0.00}{#1}}
\newcommand{\ControlFlowTok}[1]{\textcolor[rgb]{0.13,0.29,0.53}{\textbf{#1}}}
\newcommand{\DataTypeTok}[1]{\textcolor[rgb]{0.13,0.29,0.53}{#1}}
\newcommand{\DecValTok}[1]{\textcolor[rgb]{0.00,0.00,0.81}{#1}}
\newcommand{\DocumentationTok}[1]{\textcolor[rgb]{0.56,0.35,0.01}{\textbf{\textit{#1}}}}
\newcommand{\ErrorTok}[1]{\textcolor[rgb]{0.64,0.00,0.00}{\textbf{#1}}}
\newcommand{\ExtensionTok}[1]{#1}
\newcommand{\FloatTok}[1]{\textcolor[rgb]{0.00,0.00,0.81}{#1}}
\newcommand{\FunctionTok}[1]{\textcolor[rgb]{0.00,0.00,0.00}{#1}}
\newcommand{\ImportTok}[1]{#1}
\newcommand{\InformationTok}[1]{\textcolor[rgb]{0.56,0.35,0.01}{\textbf{\textit{#1}}}}
\newcommand{\KeywordTok}[1]{\textcolor[rgb]{0.13,0.29,0.53}{\textbf{#1}}}
\newcommand{\NormalTok}[1]{#1}
\newcommand{\OperatorTok}[1]{\textcolor[rgb]{0.81,0.36,0.00}{\textbf{#1}}}
\newcommand{\OtherTok}[1]{\textcolor[rgb]{0.56,0.35,0.01}{#1}}
\newcommand{\PreprocessorTok}[1]{\textcolor[rgb]{0.56,0.35,0.01}{\textit{#1}}}
\newcommand{\RegionMarkerTok}[1]{#1}
\newcommand{\SpecialCharTok}[1]{\textcolor[rgb]{0.00,0.00,0.00}{#1}}
\newcommand{\SpecialStringTok}[1]{\textcolor[rgb]{0.31,0.60,0.02}{#1}}
\newcommand{\StringTok}[1]{\textcolor[rgb]{0.31,0.60,0.02}{#1}}
\newcommand{\VariableTok}[1]{\textcolor[rgb]{0.00,0.00,0.00}{#1}}
\newcommand{\VerbatimStringTok}[1]{\textcolor[rgb]{0.31,0.60,0.02}{#1}}
\newcommand{\WarningTok}[1]{\textcolor[rgb]{0.56,0.35,0.01}{\textbf{\textit{#1}}}}
\usepackage{graphicx,grffile}
\makeatletter
\def\maxwidth{\ifdim\Gin@nat@width>\linewidth\linewidth\else\Gin@nat@width\fi}
\def\maxheight{\ifdim\Gin@nat@height>\textheight\textheight\else\Gin@nat@height\fi}
\makeatother
% Scale images if necessary, so that they will not overflow the page
% margins by default, and it is still possible to overwrite the defaults
% using explicit options in \includegraphics[width, height, ...]{}
\setkeys{Gin}{width=\maxwidth,height=\maxheight,keepaspectratio}
\setlength{\emergencystretch}{3em}  % prevent overfull lines
\providecommand{\tightlist}{%
  \setlength{\itemsep}{0pt}\setlength{\parskip}{0pt}}
\setcounter{secnumdepth}{0}
% Redefines (sub)paragraphs to behave more like sections
\ifx\paragraph\undefined\else
\let\oldparagraph\paragraph
\renewcommand{\paragraph}[1]{\oldparagraph{#1}\mbox{}}
\fi
\ifx\subparagraph\undefined\else
\let\oldsubparagraph\subparagraph
\renewcommand{\subparagraph}[1]{\oldsubparagraph{#1}\mbox{}}
\fi

% set default figure placement to htbp
\makeatletter
\def\fps@figure{htbp}
\makeatother


\title{Repreducible Reasearch Course Project 1}
\author{}
\date{\vspace{-2.5em}}

\begin{document}
\maketitle

\hypertarget{answering-questions-for-reprducible-reasarech-course-project-1.}{%
\subsection{Answering questions for Reprducible Reasarech course project
1.}\label{answering-questions-for-reprducible-reasarech-course-project-1.}}

\begin{itemize}
\tightlist
\item
  Loading and preprocessing the data:
\end{itemize}

\begin{Shaded}
\begin{Highlighting}[]
\KeywordTok{library}\NormalTok{(data.table)}
\KeywordTok{library}\NormalTok{(ggplot2)}
\NormalTok{activity <-}\StringTok{ }\KeywordTok{read.csv}\NormalTok{(}\StringTok{"activity.csv"}\NormalTok{)}
\end{Highlighting}
\end{Shaded}

\begin{itemize}
\tightlist
\item
  Reading data with data.tabel:
\end{itemize}

\begin{Shaded}
\begin{Highlighting}[]
\NormalTok{activityDT <-}\StringTok{ }\NormalTok{data.table}\OperatorTok{::}\KeywordTok{fread}\NormalTok{(}\DataTypeTok{input =} \StringTok{"activity.csv"}\NormalTok{)}
\end{Highlighting}
\end{Shaded}

\hypertarget{first-question-what-is-mean-total-number-of-steps-taken-per-day}{%
\subsection{First Question: What is mean total number of steps taken per
day?}\label{first-question-what-is-mean-total-number-of-steps-taken-per-day}}

\begin{itemize}
\tightlist
\item
  Calculating number of setps done daily:
\end{itemize}

\begin{Shaded}
\begin{Highlighting}[]
\NormalTok{Total_Steps <-}\StringTok{ }\NormalTok{activityDT[, }\KeywordTok{c}\NormalTok{(}\KeywordTok{lapply}\NormalTok{(.SD, sum, }\DataTypeTok{na.rm =} \OtherTok{FALSE}\NormalTok{)), .SDcols =}\StringTok{ }\KeywordTok{c}\NormalTok{(}\StringTok{"steps"}\NormalTok{), by =}\StringTok{ }\NormalTok{.(date)] }
\end{Highlighting}
\end{Shaded}

\begin{itemize}
\tightlist
\item
  A histogram of the total number of steps taken each day:
\end{itemize}

\begin{Shaded}
\begin{Highlighting}[]
\KeywordTok{ggplot}\NormalTok{(Total_Steps, }\KeywordTok{aes}\NormalTok{(}\DataTypeTok{x =}\NormalTok{ steps)) }\OperatorTok{+}
\StringTok{    }\KeywordTok{geom_histogram}\NormalTok{(}\DataTypeTok{fill =} \StringTok{"blue"}\NormalTok{, }\DataTypeTok{binwidth =} \DecValTok{1000}\NormalTok{) }\OperatorTok{+}
\StringTok{    }\KeywordTok{labs}\NormalTok{(}\DataTypeTok{title =} \StringTok{"Daily Steps"}\NormalTok{, }\DataTypeTok{x =} \StringTok{"Steps"}\NormalTok{, }\DataTypeTok{y =} \StringTok{"Frequency"}\NormalTok{)}
\end{Highlighting}
\end{Shaded}

\begin{verbatim}
## Warning: Removed 8 rows containing non-finite values (stat_bin).
\end{verbatim}

\includegraphics{PA1_template_files/figure-latex/unnamed-chunk-4-1.pdf}

\begin{itemize}
\tightlist
\item
  Calculate and report the mean and median of the total number of steps
  taken per day
\end{itemize}

\begin{Shaded}
\begin{Highlighting}[]
\NormalTok{Total_Steps[, .(}\DataTypeTok{Mean_Steps =} \KeywordTok{mean}\NormalTok{(steps, }\DataTypeTok{na.rm =} \OtherTok{TRUE}\NormalTok{), }\DataTypeTok{Median_Steps =} \KeywordTok{median}\NormalTok{(steps, }\DataTypeTok{na.rm =} \OtherTok{TRUE}\NormalTok{))]}
\end{Highlighting}
\end{Shaded}

\begin{verbatim}
##    Mean_Steps Median_Steps
## 1:   10766.19        10765
\end{verbatim}

\hypertarget{second-question-what-is-the-average-daily-activity-pattern}{%
\subsection{Second Question: What is the average daily activity
pattern?}\label{second-question-what-is-the-average-daily-activity-pattern}}

\begin{Shaded}
\begin{Highlighting}[]
\NormalTok{IntervalDT <-}\StringTok{ }\NormalTok{activityDT[, }\KeywordTok{c}\NormalTok{(}\KeywordTok{lapply}\NormalTok{(.SD, mean, }\DataTypeTok{na.rm =} \OtherTok{TRUE}\NormalTok{)), .SDcols =}\StringTok{ }\KeywordTok{c}\NormalTok{(}\StringTok{"steps"}\NormalTok{), by =}\StringTok{ }\NormalTok{.(interval)] }

\KeywordTok{ggplot}\NormalTok{(IntervalDT, }\KeywordTok{aes}\NormalTok{(}\DataTypeTok{x =}\NormalTok{ interval , }\DataTypeTok{y =}\NormalTok{ steps)) }\OperatorTok{+}\StringTok{ }\KeywordTok{geom_line}\NormalTok{(}\DataTypeTok{color=}\StringTok{"blue"}\NormalTok{, }\DataTypeTok{size=}\DecValTok{1}\NormalTok{) }\OperatorTok{+}\StringTok{ }\KeywordTok{labs}\NormalTok{(}\DataTypeTok{title =} \StringTok{"Avg. Daily Steps"}\NormalTok{, }\DataTypeTok{x =} \StringTok{"Interval"}\NormalTok{, }\DataTypeTok{y =} \StringTok{"Avg. Steps per day"}\NormalTok{)}
\end{Highlighting}
\end{Shaded}

\includegraphics{PA1_template_files/figure-latex/unnamed-chunk-6-1.pdf}

\begin{itemize}
\tightlist
\item
  Which 5-minute interval, on average across all the days in the
  dataset, contains the maximum number of steps?
\end{itemize}

\begin{Shaded}
\begin{Highlighting}[]
\NormalTok{IntervalDT[steps }\OperatorTok{==}\StringTok{ }\KeywordTok{max}\NormalTok{(steps), .(}\DataTypeTok{max_interval =}\NormalTok{ interval)]}
\end{Highlighting}
\end{Shaded}

\begin{verbatim}
##    max_interval
## 1:          835
\end{verbatim}

\hypertarget{imputing-missing-values}{%
\subsection{Imputing missing values}\label{imputing-missing-values}}

\begin{Shaded}
\begin{Highlighting}[]
\NormalTok{activityDT[}\KeywordTok{is.na}\NormalTok{(steps), .N ]}
\end{Highlighting}
\end{Shaded}

\begin{verbatim}
## [1] 2304
\end{verbatim}

\begin{itemize}
\tightlist
\item
  Filling in missing values with median of dataset.
\end{itemize}

\begin{Shaded}
\begin{Highlighting}[]
\NormalTok{activityDT[}\KeywordTok{is.na}\NormalTok{(steps), }\StringTok{"steps"}\NormalTok{] <-}\StringTok{ }\NormalTok{activityDT[, }\KeywordTok{c}\NormalTok{(}\KeywordTok{lapply}\NormalTok{(.SD, median, }\DataTypeTok{na.rm =} \OtherTok{TRUE}\NormalTok{)), .SDcols =}\StringTok{ }\KeywordTok{c}\NormalTok{(}\StringTok{"steps"}\NormalTok{)]}
\end{Highlighting}
\end{Shaded}

\begin{itemize}
\tightlist
\item
  Creating a new dataset that is equal to the original dataset but with
  the missing data filled in
\end{itemize}

\begin{Shaded}
\begin{Highlighting}[]
\NormalTok{data.table}\OperatorTok{::}\KeywordTok{fwrite}\NormalTok{(}\DataTypeTok{x =}\NormalTok{ activityDT, }\DataTypeTok{file =} \StringTok{"tidyData.csv"}\NormalTok{, }\DataTypeTok{quote =} \OtherTok{FALSE}\NormalTok{)}
\end{Highlighting}
\end{Shaded}

\begin{itemize}
\tightlist
\item
  Making a histogram of the total number of steps taken each day and
  Calculate and report the mean and median total number of steps taken
  per day.
\end{itemize}

\begin{Shaded}
\begin{Highlighting}[]
\NormalTok{Total_Steps <-}\StringTok{ }\NormalTok{activityDT[, }\KeywordTok{c}\NormalTok{(}\KeywordTok{lapply}\NormalTok{(.SD, sum)), .SDcols =}\StringTok{ }\KeywordTok{c}\NormalTok{(}\StringTok{"steps"}\NormalTok{), by =}\StringTok{ }\NormalTok{.(date)] }

\NormalTok{Total_Steps[, .(}\DataTypeTok{Mean_Steps =} \KeywordTok{mean}\NormalTok{(steps), }\DataTypeTok{Median_Steps =} \KeywordTok{median}\NormalTok{(steps))]}
\end{Highlighting}
\end{Shaded}

\begin{verbatim}
##    Mean_Steps Median_Steps
## 1:    9354.23        10395
\end{verbatim}

\begin{Shaded}
\begin{Highlighting}[]
\KeywordTok{ggplot}\NormalTok{(Total_Steps, }\KeywordTok{aes}\NormalTok{(}\DataTypeTok{x =}\NormalTok{ steps)) }\OperatorTok{+}\StringTok{ }\KeywordTok{geom_histogram}\NormalTok{(}\DataTypeTok{fill =} \StringTok{"blue"}\NormalTok{, }\DataTypeTok{binwidth =} \DecValTok{1000}\NormalTok{) }\OperatorTok{+}\StringTok{ }\KeywordTok{labs}\NormalTok{(}\DataTypeTok{title =} \StringTok{"Daily Steps"}\NormalTok{, }\DataTypeTok{x =} \StringTok{"Steps"}\NormalTok{, }\DataTypeTok{y =} \StringTok{"Frequency"}\NormalTok{)}
\end{Highlighting}
\end{Shaded}

\includegraphics{PA1_template_files/figure-latex/unnamed-chunk-11-1.pdf}

\hypertarget{are-there-differences-in-activity-patterns-between-weekdays-and-weekends}{%
\subsection{Are there differences in activity patterns between weekdays
and
weekends?}\label{are-there-differences-in-activity-patterns-between-weekdays-and-weekends}}

\begin{enumerate}
\def\labelenumi{\arabic{enumi}.}
\tightlist
\item
  Create a new factor variable in the dataset with two levels --
  ``weekday'' and ``weekend'' indicating whether a given date is a
  weekday or weekend day.
\end{enumerate}

\begin{Shaded}
\begin{Highlighting}[]
\CommentTok{# Just recreating activityDT from scratch then making the new factor variable. (No need to, just want to be clear on what the entire process is.) }
\NormalTok{activityDT <-}\StringTok{ }\NormalTok{data.table}\OperatorTok{::}\KeywordTok{fread}\NormalTok{(}\DataTypeTok{input =} \StringTok{"activity.csv"}\NormalTok{)}
\NormalTok{activityDT[, date }\OperatorTok{:}\ErrorTok{=}\StringTok{ }\KeywordTok{as.POSIXct}\NormalTok{(date, }\DataTypeTok{format =} \StringTok{"%Y-%m-%d"}\NormalTok{)]}
\NormalTok{activityDT[, }\StringTok{`}\DataTypeTok{Day of Week}\StringTok{`}\OperatorTok{:}\ErrorTok{=}\StringTok{ }\KeywordTok{weekdays}\NormalTok{(}\DataTypeTok{x =}\NormalTok{ date)]}
\NormalTok{activityDT[}\KeywordTok{grepl}\NormalTok{(}\DataTypeTok{pattern =} \StringTok{"Monday|Tuesday|Wednesday|Thursday|Friday"}\NormalTok{, }\DataTypeTok{x =} \StringTok{`}\DataTypeTok{Day of Week}\StringTok{`}\NormalTok{), }\StringTok{"weekday or weekend"}\NormalTok{] <-}\StringTok{ "weekday"}
\NormalTok{activityDT[}\KeywordTok{grepl}\NormalTok{(}\DataTypeTok{pattern =} \StringTok{"Saturday|Sunday"}\NormalTok{, }\DataTypeTok{x =} \StringTok{`}\DataTypeTok{Day of Week}\StringTok{`}\NormalTok{), }\StringTok{"weekday or weekend"}\NormalTok{] <-}\StringTok{ "weekend"}
\NormalTok{activityDT[, }\StringTok{`}\DataTypeTok{weekday or weekend}\StringTok{`} \OperatorTok{:}\ErrorTok{=}\StringTok{ }\KeywordTok{as.factor}\NormalTok{(}\StringTok{`}\DataTypeTok{weekday or weekend}\StringTok{`}\NormalTok{)]}
\KeywordTok{head}\NormalTok{(activityDT, }\DecValTok{10}\NormalTok{)}
\end{Highlighting}
\end{Shaded}

\begin{verbatim}
##     steps       date interval Day of Week weekday or weekend
##  1:    NA 2012-10-01        0      Monday            weekday
##  2:    NA 2012-10-01        5      Monday            weekday
##  3:    NA 2012-10-01       10      Monday            weekday
##  4:    NA 2012-10-01       15      Monday            weekday
##  5:    NA 2012-10-01       20      Monday            weekday
##  6:    NA 2012-10-01       25      Monday            weekday
##  7:    NA 2012-10-01       30      Monday            weekday
##  8:    NA 2012-10-01       35      Monday            weekday
##  9:    NA 2012-10-01       40      Monday            weekday
## 10:    NA 2012-10-01       45      Monday            weekday
\end{verbatim}

\begin{enumerate}
\def\labelenumi{\arabic{enumi}.}
\setcounter{enumi}{1}
\tightlist
\item
  Make a panel plot containing a time series plot (i.e.~𝚝𝚢𝚙𝚎 = ``𝚕'') of
  the 5-minute interval (x-axis) and the average number of steps taken,
  averaged across all weekday days or weekend days (y-axis). See the
  README file in the GitHub repository to see an example of what this
  plot should look like using simulated data.
\end{enumerate}

\begin{Shaded}
\begin{Highlighting}[]
\NormalTok{activityDT[}\KeywordTok{is.na}\NormalTok{(steps), }\StringTok{"steps"}\NormalTok{] <-}\StringTok{ }\NormalTok{activityDT[, }\KeywordTok{c}\NormalTok{(}\KeywordTok{lapply}\NormalTok{(.SD, median, }\DataTypeTok{na.rm =} \OtherTok{TRUE}\NormalTok{)), .SDcols =}\StringTok{ }\KeywordTok{c}\NormalTok{(}\StringTok{"steps"}\NormalTok{)]}
\NormalTok{IntervalDT <-}\StringTok{ }\NormalTok{activityDT[, }\KeywordTok{c}\NormalTok{(}\KeywordTok{lapply}\NormalTok{(.SD, mean, }\DataTypeTok{na.rm =} \OtherTok{TRUE}\NormalTok{)), .SDcols =}\StringTok{ }\KeywordTok{c}\NormalTok{(}\StringTok{"steps"}\NormalTok{), by =}\StringTok{ }\NormalTok{.(interval, }\StringTok{`}\DataTypeTok{weekday or weekend}\StringTok{`}\NormalTok{)] }
\KeywordTok{ggplot}\NormalTok{(IntervalDT , }\KeywordTok{aes}\NormalTok{(}\DataTypeTok{x =}\NormalTok{ interval , }\DataTypeTok{y =}\NormalTok{ steps, }\DataTypeTok{color=}\StringTok{`}\DataTypeTok{weekday or weekend}\StringTok{`}\NormalTok{)) }\OperatorTok{+}\StringTok{ }\KeywordTok{geom_line}\NormalTok{() }\OperatorTok{+}\StringTok{ }\KeywordTok{labs}\NormalTok{(}\DataTypeTok{title =} \StringTok{"Avg. Daily Steps by Weektype"}\NormalTok{, }\DataTypeTok{x =} \StringTok{"Interval"}\NormalTok{, }\DataTypeTok{y =} \StringTok{"No. of Steps"}\NormalTok{) }\OperatorTok{+}\StringTok{ }\KeywordTok{facet_wrap}\NormalTok{(}\OperatorTok{~}\StringTok{`}\DataTypeTok{weekday or weekend}\StringTok{`}\NormalTok{ , }\DataTypeTok{ncol =} \DecValTok{1}\NormalTok{, }\DataTypeTok{nrow=}\DecValTok{2}\NormalTok{)}
\end{Highlighting}
\end{Shaded}

\includegraphics{PA1_template_files/figure-latex/unnamed-chunk-13-1.pdf}

\end{document}
